\documentclass[usenames,dvipsnames]{beamer}

\usepackage{xcolor}
\usepackage[utf8]{inputenc}
\usepackage{listings}
\usepackage{tikz}
\usepackage[style=alphabetic,giveninits=true,terseinits=true]{biblatex}
\usepackage{graphicx}

\usetikzlibrary{matrix, positioning, decorations.pathreplacing, overlay-beamer-styles}

\definecolor{tumblue}{RGB}{0,101,189}
\definecolor{dkgreen}{rgb}{0,0.6,0}
\definecolor{mauve}{rgb}{0.58,0,0.82}
\lstset{
    basicstyle=\footnotesize\ttfamily,%
    columns=fullflexible,%
    keywordstyle=\bfseries\color{blue},%
    commentstyle=\color{dkgreen},%
    stringstyle=\color{mauve},%
}

\DeclareFieldFormat{url}{\href{#1}{\resizebox{!}{2ex}{\lower3pt\hbox{\pgfuseimage{beamericononline}}}}}
\addbibresource{slides.bib}

\setbeamerfont{footnote}{size=\tiny}
\renewcommand\footnoterule{}
\usecolortheme{orchid}
\setbeamertemplate{navigation symbols}{}
\setbeamertemplate{sidebar right}{% also implies no navbar
    \llap{\tikz\fill[tumblue,scale=.1] (0,0) ++(-5cm,-5cm) ++(-10cm,0)
        +(0cm,0cm) -- +(4cm,0cm) -- +(4cm,-4cm) -- +(5cm,-4cm) -- +(5cm,0cm) --
        +(10cm,0cm) -- +(10cm,-5cm) -- +(9cm,-5cm) -- +(9cm,-1cm) --
        +(8cm,-1cm) -- +(8cm,-5cm) -- +(7cm,-5cm) -- +(7cm,-1cm) --
        +(6cm,-1cm) -- +(6cm,-5cm) -- +(3cm,-5cm) -- +(3cm,-1cm) --
        +(2cm,-1cm) -- +(2cm,-5cm) -- +(1cm,-5cm) -- +(1cm,-1cm) --
        +(0cm,-1cm) -- cycle;}
    % Slide number in triangle.
    \vfill\llap{\tikz\path[fill=tumblue,inner sep=3pt] (0,0) -- ++(-1cm,0) -- +(1cm,1cm) -- cycle node[anchor=south east] {\usebeamerfont{footline}\color{white}\bfseries\insertframenumber};}%
% Variant without triangle
%\vfill\llap{\usebeamerfont{footline}\usebeamercolor[fg]{footline}\insertframenumber\hskip3pt}\vskip3pt%
}

\graphicspath{ {./images/} }

\title{CoPaaS - Compiler-oriented Programming as a Service}
\author{Simon Bußmann \and Johannes Maier}
\institute{School of Computation, Information, and Technology\\Technical University of Munich}
\date{Future}

\begin{document}

    \begin{frame}
        \titlepage
    \end{frame}

%    \begin{frame}{Organization}
%        \begin{itemize}
%            % Don't forget to adjust the \addbibresource above.
%            \item Citation example, if you really need it\footfullcite{nielson2003marching}
%        \end{itemize}
%    \end{frame}

    \begin{frame}{The Story}

    \end{frame}

    \begin{frame}{x86-64 Instruction Encoding Basics}
        \begin{columns}
            \begin{column}{0.35\textwidth}

                Pars of an instruction:
                \begin{itemize}
                    \item \textcolor<2>{blue}{Prefixes}
                    \item Opcode
                    \item \textcolor<3-5>{blue}{ModR/M}
                    \item SIB
                    \item Displacement
                    \item Immediate
                \end{itemize}
            \end{column}
            \begin{column}{0.65\textwidth}
                \begin{tikzpicture}[scale=0.75]

                    % REX.W prefix
                    \matrix (rex) at (0,0) [
                    matrix of nodes,
                    nodes={draw, minimum size=18pt, anchor=center},
                    column sep=-\pgflinewidth,
                    row sep=-\pgflinewidth,
                    ampersand replacement=\&
                    ] {
                        0 \& 1\& 0 \& 0 \& \textcolor<2>{blue}{W} \& \textcolor<2>{blue}{R} \& X \& \textcolor<2>{blue}{B} \\
                    };

                    % ModR/M byte
                    \matrix (modrm) [
                        below=20pt of rex,
                        matrix of nodes,
                        nodes={draw, minimum size=18pt, anchor=center},
                        column sep=-\pgflinewidth,
                        row sep=-\pgflinewidth,
                        ampersand replacement=\&
                    ] {
                        \textcolor<3>{blue}{1} \& \textcolor<3>{blue}{1} \& \textcolor<4>{blue}{0} \& \textcolor<4>{blue}{1} \& \textcolor<4>{blue}{0} \& \textcolor<5>{blue}{0} \& \textcolor<5>{blue}{0} \& \textcolor<5>{blue}{0} \\
                    };

                    % Labels
                    \node[anchor=east] at (rex.west) {REX.W};
                    \node[anchor=east] at (modrm.west) {ModR/M};

                    % Brackets for ModR/M parts
                    \draw[semithick, decorate, decoration={brace, amplitude=10pt, mirror, raise=4pt}, alt=<3>{blue}{black}] (modrm-1-1.south west) -- (modrm-1-2.south east) node[midway, below=14pt] {\textcolor<3>{blue}{Modifier}};
                    \draw[semithick, decorate, decoration={brace, amplitude=10pt, mirror, raise=4pt}, alt=<4>{blue}{black}] (modrm-1-3.south west) -- (modrm-1-5.south east) node[midway, below=14pt, xshift=-4pt] {\textcolor<4>{blue}{Reg}};
                    \draw[semithick, decorate, decoration={brace, amplitude=10pt, mirror, raise=4pt}, alt=<5>{blue}{black}] (modrm-1-6.south west) -- (modrm-1-8.south east) node[midway, below=14pt] {\textcolor<5>{blue}{RM}};

                \end{tikzpicture}
            \end{column}
        \end{columns}
    \end{frame}

    \begin{frame}{The Bug}
        \only<2-3> {
            \begin{figure}
                \centering
                \includegraphics[scale=0.65]{bug}
            \end{figure}
        }
        \visible<3>{No, not you...}
        \only<4->{
            \begin{itemize}
                \item Faulty encoding of instructions using \lstinline{r[8-15]}
                \item[$\Rightarrow$] Register ids are 4bit long
                \item Example: encode ModR/M for \lstinline{mov r8, r9}
                \item Register ids: $\text r8 = 0\text b1000, \text r9 = 0\text b1001$
                \[\text{Mod R/M} = \textcolor<4>{blue}{0\text{b}1100\_0000} + \textcolor<5>{blue}{(\text{reg2\_id} \ll 3)} + \textcolor<6>{blue}{\text{reg1\_id}}\]
            \end{itemize}

            \begin{center}
                \begin{tikzpicture}
                    \matrix (modrm) [
                        below=20pt of rex,
                        matrix of nodes,
                        nodes={draw, minimum size=18pt, anchor=center},
                        column sep=-\pgflinewidth,
                        row sep=-\pgflinewidth,
                        ampersand replacement=\&
                    ] {
                        \textcolor<4>{blue}{\textcolor<5>{red}{\alt<5->{0}{1}}} \&
                        \textcolor<4>{blue}{\textcolor<5>{red}{\alt<5->{0}{1}}} \&
                        \textcolor<5>{blue}{0} \&
                        \textcolor<5>{blue}{\textcolor<6>{red}{\alt<6->{1}{0}}} \&
                        \textcolor<5>{blue}{\textcolor<6>{red}{\alt<5>{1}{0}}} \&
                        \textcolor<6>{blue}{0} \&
                        \textcolor<6>{blue}{0} \&
                        \textcolor<6>{blue}{0} \\
                    };
                \end{tikzpicture}
            \end{center}
        }
    \end{frame}

    \begin{frame}{Exploit Primitives}
        But, what can we do with that?
        % TODO: describe primitives further
        \begin{itemize}
            \item Arbitrary write: \lstinline{mov [reg1], reg2}
            \only<1-2> {
                \begin{itemize}
                    \item Use \lstinline{COPY} to generate \lstinline{mov}, overflow into modifier bits
                    \item Register id constraints: $\text{reg1\_id} = 0\text{b}0\text{xxx}, \text{reg2\_id}=0\text{b}1\text{xxx}$
                    \item[$\Rightarrow$] But can rename registers freely
                    \item Example: \texttt{mov [\textcolor{ForestGreen}{rax}], \textcolor{blue}{r8}}
                \end{itemize}

                \begin{center}
                    \begin{tikzpicture}
                        \matrix (modrm) [
                            below=20pt of rex,
                            matrix of nodes,
                            nodes={draw, minimum size=18pt, anchor=center},
                            column sep=-\pgflinewidth,
                            row sep=-\pgflinewidth,
                            ampersand replacement=\&
                        ] {
                            \textcolor<1>{black}{\textcolor<2>{red}{\alt<2>{0}{1}}} \&
                            \textcolor<1>{black}{\textcolor<2>{red}{\alt<2>{0}{1}}} \&
                            \textcolor<2>{blue}{0} \&
                            \textcolor<2>{blue}{0} \&
                            \textcolor<2>{blue}{0} \&
                            \textcolor<2>{ForestGreen}{0} \&
                            \textcolor<2>{ForestGreen}{0} \&
                            \textcolor<2>{ForestGreen}{0} \\
                        };
                    \end{tikzpicture}
                \end{center}
            }
            \item Arbitrary read: \lstinline{mov reg1, [reg2]}
            \only<3>{
                \begin{itemize}
                    \item Actually: \lstinline{mov reg1, 0; add reg1, [reg2]}
                    \item Use same trick as for write
                    \item There exists multiple opcodes for add\footnote{Also for most other standard x86 operations.}, for example:
                    \item Opcode $= 0x1$: \lstinline{mov [reg1], reg2}
                    \item Opcode $= 0x3$: \lstinline{mov reg1, [reg2]}
                \end{itemize}
            }
            \item Stack access
            \only<4> {
                \begin{itemize}
                    \item Another overflow, but this time into \lstinline{reg2_id}
                    \item Can create \lstinline{reg2_id + 1}, using reg1\_id = $0\text{b}1$xxx
                    \item Id of \lstinline{rsp} is $0\text{b}0100$
                    \item[$\Rightarrow$] For \lstinline{COPY G, B} is actually \lstinline{mov r8, rsp} generated
                    \item Combined allow those primitives unlimited access to the stack
                \end{itemize}
            }
        \end{itemize}
    \end{frame}

    \begin{frame}{The Exploit}
        Steps:
        \begin{enumerate}
            \item Leak activation key
            \item Escape seccomp jail
            \item Execute \lstinline{system("/bin/sh")}
            \item Profit
        \end{enumerate}
    \end{frame}

\end{document}
